\section{Μεθοδολογία}

\graphicspath{{Chapter3/Figs/Vector/}}

\begin{frame}
  \frametitle{Εισαγωγή εξισώσεων και εικόνων}
  \framesubtitle{Ανάλυση τριγώνου - Ο αλγόριθμος του Frank}
  \label{fr3:frank_tr}
  \begin{columns}
    \begin{column}{.65\textwidth}
    %\vskip.1cm
      {\small Η διαφορά της γωνίας σε δύο χρονικές στιγμές:}
      \[ \delta\phi_\alpha=\delta \theta_\beta - \delta \theta_\gamma \]
 \vskip.1cm
{\small Οι συνιστώσες της διάτμησης:}
  \begin{tiny}
  \begin{align*}
  \gamma_1=\frac{\sin\left ( \theta_\gamma + \theta_\alpha \right )\left ( \delta\phi_\alpha/\sin\alpha_\alpha \right ) - \sin\left ( \theta_\beta + \theta_\gamma \right )\left ( \delta\phi_\beta/\sin\alpha_\beta \right )}{\sin\phi_\gamma} \\
  \gamma_2=\frac{\cos\left ( \theta_\gamma + \theta_\alpha \right )\left ( \delta\phi_\alpha/\sin\alpha_\alpha \right ) - \cos\left ( \theta_\beta + \theta_\gamma \right )\left ( \delta\phi_\beta/\sin\alpha_\beta \right )}{\sin\phi_\gamma}
  \end{align*}
  \end{tiny}
   {\small Η τιμή της ολικής διάτμησης και το αζιμούθιο των κύριων αξόνων της έλλειψης της ανηγμένης παραμόρφωσης:}
    \end{column}
    \begin{column}{.34\textwidth}
      \centering
          \adjincludegraphics[width=.98\linewidth, valign=t]{FrankTR.jpg}
          
          {\tiny \parencite{Frank1966}}
    \end{column}
  \end{columns}
\vskip-.5cm  
\begin{columns}
  \begin{column}{.3\textwidth}
    \begin{small}
  \vskip-1cm
      \begin{equation*}
        E=\begin{bmatrix}
        \varepsilon_{11} & \varepsilon_{12}\\ 
        \varepsilon_{21} & \varepsilon_{22}
        \end{bmatrix}
      \end{equation*}
    \end{small}
  \end{column}
  \begin{column}{.03\textwidth}
    \vskip-1cm
    \textbf{$\Rightarrow$}
  \end{column}
  \begin{column}{.3\textwidth}
    \begin{scriptsize}
      \begin{align*}
        \gamma _1=\varepsilon_{11} - \varepsilon_{22}, \\
        \gamma _2=\varepsilon_{12} + \varepsilon_{21}, \\ 
        \omega = \frac{1}{2} \left (\varepsilon_{12} - \varepsilon_{21}  \right )\\
      \end{align*}
    \end{scriptsize}
  \end{column}
   \begin{column}{.03\textwidth}
     \vskip-1cm
    \textbf{$\Rightarrow$}
  \end{column}
  \begin{column}{.3\textwidth}
  
%   \begin{tcolorbox}[height=1.5cm, width=2.5cm]
%     \begin{scriptsize}
%       \begin{align*} 
%         \gamma = \sqrt{{\gamma_{1}}^{2}+{\gamma_{2}}^{2}}\\
%         \tan 2\psi = \frac{\gamma_1}{\gamma_2} \\
%       \end{align*}
%     \end{scriptsize}
%     \end{tcolorbox}
    
% \begin{varblock}[2.5cm]{}
  \centering
    \begin{scriptsize}
      \begin{align*} 
        \gamma = \sqrt{{\gamma_{1}}^{2}+{\gamma_{2}}^{2}}\\
        \tan 2\psi = \frac{\gamma_1}{\gamma_2} \\
      \end{align*}
    \end{scriptsize}
% \end{varblock}
  \end{column}
\end{columns}


\end{frame}
\note{} % Add notes for this slide